\documentclass[12pt, letterpaper]{article}
\usepackage[serbian]{babel}

\title{AI koji generiše sliku na osnovu teksta}
\author{Budimir Nkola 178/2021\\Trajković Miljan 354/2022\\Cvejić Miloš 346/2021\\Bajić Bogdan 122/2021}

\begin{document}


\maketitle
Ovo je test!

\begin{abstract}
\end{abstract}

\begin{tableofcontents}
\end{tableofcontents}

\pagebreak
\section{Etički problemi sa veštačko inteligentnim generatorima slika}

AI generatori slika mogu doprineti diskriminaciji tako što oponašaju štetne stereotipe koje je AI prikupio u kolekcijama podataka, koji sami imaju predrasude iz svakodnevnog života.

To se može korigovati zabranjivanjem određenih reči ili nasumičnim dodavanjem prefiksa kao što su: žena/muškarac/trans, crna/žuta/braon osoba, kvir\footnote{kvir (engl. queer) - naziv za celokupnu homoseksualnu, biseksualnu, transrodnu i interseksualnu zajednicu kao i heteroseksualne osobe koje sebe vide ili žive svoj život van heteropatrijarhalnih normi.} itd.). Ali nekada automatsko dodavanje dodatnih reči može negativno uticati umesto da reši problem. Npr. dodavanje reči žena za neke termine koji se koriste za osobe svih polova ili dodavanje reči crnac kada se spominju bande/kartele.

Još jedan problem je mogućnost veštačke inteligencije da generiše lažne informacije. To je upravo razlog zašto DALL-E ne dozvoljava generisanje slika sa poznatim osobama, ali lažne informacije ne utiču samo na poznate ljude. Npr. ljudi se mogu svetiti svojim bivšim partnerima ili članovima svoje porodice.

Ali možda najstrašnije stvar jeste da ova tehnologija može u potpunosti uništiti naše poverenje u slike kao verodostojan dokaz. Zbog straha da su slike lažne, jer generator slika nam omogućava da kreiramo fotorealistične slike.\\
Opasnost od ovog scenarija se može umanjiti korišćenjem tehnologija koje imaju sposobnost detektovanja lažnih slika. Takođe mogli bismo zahtevati da veštački generisane slike imaju neku oznaku, pogotovo kada su u pitanju fotorealistične slike.



\end{document}
