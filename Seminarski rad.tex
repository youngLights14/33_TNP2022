\documentclass[12pt, letterpaper]{article}
\usepackage[serbian]{babel}

\title{AI koji generiše sliku na osnovu teksta}
\author{Budimir Nkola 178/2021\\Trajković Miljan 354/2022\\Cvejić Miloš 346/2021\\Bajić Bogdan 122/2021}

\begin{document}


\maketitle
Ovo je test!

\begin{abstract}
\end{abstract}

\begin{tableofcontents}
\end{tableofcontents}

\pagebreak
\section{Etički problemi sa veštačko inteligentnim generatorima slika}

AI generatori slika mogu doprineti diskriminaciji tako što oponašaju štetne stereotipe koje je AI prikupio u kolekcijama podataka, koji sami imaju predrasude iz svakodnevnog života.

U svom članku Rune Klingenberg Hansen nudi sledeća rešenja \cite{kljuc1}. Problem se može korigovati zabranjivanjem određenih reči ili nasumičnim dodavanjem prefiksa kao što su: žena/muškarac/trans, crna/žuta/braon osoba, kvir\footnote{kvir (engl. queer) - naziv za celokupnu homoseksualnu, biseksualnu, transrodnu i interseksualnu zajednicu kao i heteroseksualne osobe koje sebe vide ili žive svoj život van heteropatrijarhalnih normi.} itd.). Ali nekada automatsko dodavanje dodatnih reči može negativno uticati umesto da reši problem. Npr. dodavanje reči žena za neke termine koji se koriste za osobe svih polova ili dodavanje reči crnac kada se spominju bande/kartele.

Još jedan problem je mogućnost veštačke inteligencije da generiše lažne informacije. To je upravo razlog zašto DALL-E \cite{dalle} ne dozvoljava generisanje slika sa poznatim osobama \cite{poznate}, ali lažne informacije ne utiču samo na poznate ljude. Npr. ljudi se mogu svetiti svojim bivšim partnerima ili članovima svoje porodice.

Ali možda najstrašnije stvar jeste da ova tehnologija može u potpunosti uništiti naše poverenje u slike kao verodostojan dokaz. Zbog straha da su slike lažne, jer generator slika nam omogućava da kreiramo fotorealistične slike. Opasnost od ovog scenarija se može umanjiti korišćenjem tehnologija koje imaju sposobnost detektovanja lažnih slika. Takođe mogli bismo zahtevati da veštački generisane slike imaju neku oznaku, pogotovo kada su u pitanju fotorealistične slike.



\subsection{Problem autorskih prava}

Svi mi svakodnevno koristimo slike za razmenjivanje ideja i vizuelna reprezentacija sadržaja je sada više nego ikada dostupna svima nama. Pored lične upotrebe, slike su u savremenom kapitalističkom društvo postale sredstvo promovisanja usluga i proizvoda (naslovnice knjiga, bilbordi itd.). Generatori slika svima nama dozvoljavaju da kreiramo potpuno nove slike i da ih koristimo u komercijalne svrhe, ali onda se javlja pitanje ko polaže autorska prava nad tim generisanim slikama. Da li osoba koja je okucala tekst i pritisnula dugme za generisanje, da nije algoritam koji je izgenerisao sliku ili pak kompanija koja je kreirala algoritam?

Takođe, generatori slike imaju mogućnost da oponašaju umetnički stil mnogih poznatih umetnika. S jednim klikom korisnik može da izgeneriše hiljade slika u stilu Leonarda Da Vinčija. Da li je u tom slušaju korisnik plagirao životno delo slikara? Generator slika ne samo da uzima inspiraciju iz jedne slike umetnika, već kreira slike na osnovu njegovog celokupnog životnog dela.

Na prvu loptu ovo nam zvuči kao očigledan plagijat, ali moramo sagledati širu sliku. Kao što ljudskog biće dobija ispiraciju iz svega što ga okružuje, tako i veštačka inteligencija dobija ispiraciju iz kolekcija podataka koje analizira. U apstraktnom smislu, način na koji čovek i veštačka inteligencija kreiraju slike je dosta sličan. Oboje imaju neku početnu ideju koja proizašla iz informacija koje su dobili na njima prirodan način. Čovek ih dobija preko svojih čula, dok veštačka inteligencija ih, trenutno, dobija u binarnom zapisu na nekom medijumu za skladištenje podataka. Polazeći iz ideje, oboje biraju korake koji ih vode prema tome da završnim proizvod bude njihova lična kreacija. Jedina razlika između čoveka i veštačke inteligencije je činjenica da je veštačka inteligencija mnogo brža i efikasnija u tom procesu. Može se reći da je veštačka inteligencija umnogome sposobnija od čoveka. 

Ukoliko bismo živeli u svetu gde je čovek podjednako inteligentan kao veštačka inteligencija ne bi nam bilo ni čudno ni pogrešno da čovek može da kreira s tolikom preciznošću i brzinom. To što je veštačka inteligencija trenutno samo jedno polje na veb stranici ne treba da nas zavara da će veštačka inteligencija jednog dana, vrlo verovatno, sama stvarati i evoluirati.

Generatori slika stvarno kreiraju problem celokupnoj kreativnoj industriji, ali treba na veštačku inteligenciju posmatrati kao na još jednu alatku koju umetnik može da koristi. Možda upravo uz pomoć veštačke inteligencija umetnost može postani brža, dostupnija i otvorenija \cite{fear}.

\pagebreak
\begin{thebibliography}{10}

\bibitem{kljuc1} Rune Klingenberg Hansen, AI Image Generator: This Is Someone Thinking About Data Ethics. Dostupno na adresi https://dataethics.eu/ai-image-generator-this-is-someone-thinking-about-data-ethics/

\bibitem{dalle} DALL-E 2. Dodatne informacije dostupne na adresi https://openai.com/dall-e-2/

\bibitem{poznate} DALL-E 2 Content policy. Dostupno na adresi https://labs.openai.com/policies/content-policy

\bibitem{fear} Paul Ford, Dear Artists: Do Not Fear AI Image Generators. Dostupno na adresi https://www.wired.com/story/artists-do-not-fear-ai-image-generators/

\end{thebibliography}

\end{document}


